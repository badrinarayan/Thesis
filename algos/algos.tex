%!TEX root = ../dissertation.tex
\chapter{Algorithms}
\label{chap:algos}

As alluded to in Chapter 1, the problems of atomic norm decomposition and
regularization have linear and quadratic objectives can be efficiently computed
provided there is an efficient way to test membership in the constraint sets.
The constraint sets of the primal and dual problems are the sublevel sets of the
atomic and the dual atomic norm. Therefore, it is sufficient to develop
efficient characterizations of the atomic norm ball.

For the special case of Fourier measurements, the atomic norm ball has
semidefinite characterizations which we derive in this chapter. The positive
case is classical and comes from moment theory and the dual theory of positive
polynomials. We will review the results for the positive case and provide the
proofs for completeness. We will also derive the semidefinite characterization
for the more general complex case using these results.

For the problem of system identification, there are no tractable semidefinite
formulations. While it is possible to develop a sequence of semidefinite
relaxations, we instead describe how we can use a discretization approach for
the atomic soft thresholding (AST) problem described in Chapter~\ref{chap:ast}
and this essentially reduces to solving a Lasso problem on an overcomplete grid.
Our proofs for discretized atomic soft thresholding (DAST) demonstrate why Lasso
is often successful even for off-grid data.


\subsection*{Summary and Organization of this chapter} % (fold)
\label{sub:main_results}
Our goal in this chapter is to develop an algorithm to solve the Atomic Soft
Thresholding (AST) problem:
\begin{align}
\begin{split}
\minimize_x & \frac{1}{2}\vnorm{y-x}_2^2 + \tau \vnorm{x}_\A,
\end{split}	
\end{align}
and its dual:
\begin{align}
\begin{split}
\maximize_x & \frac{1}{2}\left(\vnorm{y}_2^2 - \vnorm{y-\tau q}_2^2\right)\\
\text{subject to } & \vnorm{q}_\A^* \leq 1. 
\end{split}
\end{align}
To this end, we will develop computable characterizations of the atomic norm and its dual. First, let us recall the definition of the trigonometric monomial $a(f),$ defined in Chapter~\ref{chap:linespect} which form the basic atoms for Line Spectral Estimation:
{\footnotesize
\[
a(f) = \begin{pmatrix}
	1\\
	e^{i 2\pi f}\\
	\vdots\\
	e^{i2\pi f (n-1)}
\end{pmatrix}
\]
}
The first part of this chapter gives a computable algebraic characterization of the atomic norm balls of the following atomic sets derived from these atoms:
\begin{align}
\A_+ &= \setof{a(f)}{f \in \mathbb{T}}\\
\A   &= \setof{a(f) \exp(i 2\pi \phi)}{f, \phi \in \mathbb{T}}
\end{align}
The first set $\A_+$ is a one dimensional manifold called the trigonometric
moment curve and the second set $\A$ is the two dimensional manifold
corresponding to a phase-symmetric trigonometric moment curve. The unit norm
ball of the atomic sets are precisely the convex hulls of the atomic sets. In
Section~\ref{sec:preliminaries}, we will look at these sets. We first show the
classical characterization of the conical hull of $\A_+$, called the moment
cone, corresponds to allowable observations for the the case of line spectral
estimation with nonnegative amplitudes, or alternatively valid trigonometric
moments of a positive measure on the torus $\mathbb{T}$. 

We provide semidefinite characterizations of the atomic sets in Section~\ref{sec:sdp_for_trigonometric_moments}. As a straightforward consequence of classical result, we can write down a semidefinite
characterization of $\conv(\A_+)$:
\begin{theorem}
\label{thm:positive-linespect-sdp}
Suppose $x = (x_0 \cdots x_{n-1})^T \in \C^n.$ Then, 
	\[
		\vnorm{x}_{\A_{+}} = \begin{cases}
			x_0, & T x \succeq 0,\\
			+\infty, & \text{otherwise.}
		\end{cases}
	\]
\end{theorem}

However, The semidefinite characterization of $\conv{\A}$ requires some
computation.The characterization of moment cone allows us to describe the convex
hull of the general atomic moments~$\conv(\A)$ and we will derive the following
semidefinite characterization:

\begin{theorem}\label{thm:sdp-char}
	For $x\in \C^n$,
	\begin{equation}
	\label{eq:atomic_sdp}
	\|x\|_{\mathcal{A}} = \inf\setof{\tfrac{1}{2n} \tr( T_n(u)) +\tfrac{1}{2} t}{\begin{bmatrix}
T_n(u) & x\\
x^* & t
\end{bmatrix} \succeq 0}.
	\end{equation}
\end{theorem}

The characterization of atomic norm ball of $\A$ can also be derived by working on the dual problem. In fact, for any $x,$ the optimization problem
\begin{equation}
	\label{eq:dual-form-atomic-norm}
	\begin{aligned}
		\minimize_q~ & \vabs{q}{x} &&\\
		\text{such that } & \vnorm{q}_\A^*	 \leq 1.	&&
	\end{aligned}
\end{equation}
is the dual characterization which has an optimum value $\vnorm{x}_\A$. For the atomic set of general trigonometric moments, the constraint set in Equation \eqref{eq:dual-form-atomic-norm}
\begin{align}
	\setof{q \in \C^n}{\vnorm{q}_\A^* \leq 1} &
	= \setof{q \in \C^n}{\sup_{a \in \A}\vabs{q}{a} \leq 1} \\
	&= \setof{q \in \C^n}{\sum_{j=0}^{n-1}{|q_j \exp(i 2\pi j f)|} \leq 1, \text{ for all } f \in \mathbb{T}}
\end{align}
is the set of complex trigonometric polynomials with a maximum modulus of $1$
and is characterized by the Bounded Real Lemma. 

When there are no efficient characterization, we may resort to a discretized
approach which may be thought of as an approximation to the atomic norm defined
by equation~\eqref{eq:dual-form-atomic-norm} by relaxing the semi-infinite
program to a finite set of inequalities. We show convergence rates for this
approximation for the line spectral estimation problem and the system
identification problem. In this case, the atomic soft thresholding problem can
be approximated by solving a Lasso problem on an overcomplete grid.

% subsection main_results (end)

\section{Preliminaries} % (fold)
\label{sec:preliminaries}

The observations in a line spectral estimation problem may be regarded as
trigonometric moments of a measure. 

\begin{definition}[Trigonometric Moments]
Given a measure $\mu$ defined on the torus $\mathbb{T}$, the $m$th
trigonometric moment is defined by
\begin{equation}
	\label{eq:trig-moment-measure}
	x_m = \int_\mathbb{T} e^{i 2 \pi k f} \mu ( d f)
\end{equation}
\end{definition}

By a simple reparametrization, the $m$th trigonometric moment may be described
as complex moments $\int_\mathbb{T} z^m \mu(dz)$ when we regard $\mu$ as defined
on the unit circle. In this section, we will review two classical results. 

The first theorem, due to Herglotz gives a complete characterization of the
infinite sequence of Trigonometric moments for positive measures. Before we
state the theorem, we will need a bit of notation. Define the map
$T_n:\mathbb{C}^n \rightarrow \mathbb{C}^{n\times n}$ which creates a Hermitian
Toeplitz matrix out of its input, that is

\[
T_n(x)= \left[
\begin{array}{ccccc} x_1 & x_2 & \ldots & x_n\\ 
x^*_2 & x_1  & \ldots & x_{n-1}\\
 \vdots & \vdots & \ddots & \vdots\\
 x^*_n & x^*_{n-1}  & \ldots & x_1
 \end{array}\right]
\]

\begin{theorem}[Herglotz Theorem]\label{thm:herglotz}
	The sequence of complex numbers $\{x_m\}_{m=-\infty}^{\infty}$ are
trigonometric moments of a positive Borel measure $\mu$ on $\mathbb{T}$ if and
only if the sequence $\{x_m\}_{m=-\infty}^{\infty}$ is positive definite.
\end{theorem}

\begin{proof}

A sequence $\set{x_m}$ is positive definite if for every $n \in \N$, and every
sequence $(c_1, \ldots, c_n)$ of $n$ complex numbers, $\sum_{k,l=1}^n x_{k-1}
c_k c_l^* \geq 0.$ Using our notation for the Toeplitz map, we can simply write
this condition as $T_n x \succeq 0$ for every $n \in \N.$

Suppose indeed the set $\set{x_m}$ is a sequence of trigonometric moments
corresponding to some measure $\mu > 0$ on $\mathbb{T}.$ Then, for any $n \in
\N,$

\begin{align*}
	\sum_{k,l=1}^n x_{k-l} c_k c_l^* &= \sum_{k,l=1}^n c_k c_l^* \int \exp(i 2 \pi (k-l) t ) \mu (dt)\\
	&= \int \left(\sum_{k=1}^n c_k \exp(i 2 \pi k t) \right) \left(\sum_{k=1}^n c_l \exp(i 2 \pi l t) \right)^* \mu (dt) \\
	&= \int \left|\sum_{k=1}^n c_k \exp(i 2 \pi k t) \right|^2 \mu (dt) \geq 0.
\end{align*}

Conversely, if the sequence $\set{x_m}$ is positive semidefinite, then for any $t \in \R,$  we have
\[
X_n(t) = \sum_{k,l=1}^n x_{k-l} e^{i 2 \pi (k-l) t} \geq 0
\]
But, we have
\[
	X_n(t) = \sum_{k=-(n-1)}^{(n-1)} \left( 1 - \frac{|k|}{n}\right) x_k \exp(i 2\pi k t)
\]
By Fourier Series theory, we have that for every $|k| \leq n,$

\begin{equation}
	\label{eq:moment-sequence-limit}
	\left( 1 - \frac{|k|}{n} \right) x_k = \int_0^1 X_n(t) \exp(i 2\pi k t ) d t
\end{equation}

Define a sequence of measures $\mu_n$ given by
\[
	\mu_n(B) = \int_0^1 X_n(t) d t
\]

for every Borel subset $B \subset [0,1].$ The sequence of measures $\set{\mu_n}$
are tight and therefore by Helly selection theorem, there exists a subsequence
$\mu_{n_k}$ which converges weakly to a measure $\mu$ on $\mathbb{T}.$ Finally, using~\ref{eq:moment-sequence-limit} we conclude that $\set{x_k}$ is the sequence of moments for the measure $\mu.$
\end{proof}

We are interested in \emph{discrete} positive measures, i.e., measures composed
of a finite number of atoms at $f_1, \ldots, f_l$, so we can write

\[
\mu(f) = \sum_{l=1}^k c_l \xi_{f_l}
\]

where $\xi_f$ denotes the point measure at $f$ and $\set{c_l} > 0$ are the
amplitudes. The first $n$ moments of such a measure concentrated on $k$ atoms
corresponds to a $k$ simple combination of atomic moment sequences $a(f).$ In
fact,

\begin{align*}
	x_m &= \int_\mathbb{T} e^{i 2 \pi k f} \sum_{l=1}^k c_l \xi_{f_l}(d f)\\
	&= \sum_{l=1}^k c_l e^{i 2 \pi f_l m}.
\end{align*}

Then the moment vector
$x = \begin{pmatrix}x_0 & \cdots & x_{n-1}\end{pmatrix} \in \C^n$ is given by
\[
x = \sum_{l=1}^k c_l a(f_l)
\]

Our second classical result due to Caratheodory and Toeplitz gives conditions
under which one can find a discrete measure corresponding to a partially
observed sequence of moments.

\begin{theorem}[Caratheodory-Toeplitz theorem,~\cite{toeplitz1911theorie,caratheodory1911zusammenhang,caratheodory1911variabilitatsbereich}]
\label{thm:caratheodory-toeplitz}
	$x \in \C^n$ corresponds to the first $n$ trigonometric moments of a measure $\mu$ (i.e., $x \in \cone(\A_+)$) only if and only if $T x \succeq 0.$ Furthermore, if $k = \rank(T x)$, there exists positive numbers $c_1, \ldots, c_k$ and $f_1, \ldots, f_k \in \mathbb{T}$ such  that
\[
	\mu = \sum_{l=1}^k c_l \delta(f - f_l)
\]
so that
\begin{align*}
	x &= \int_{\mathbb{T}} a(f) \mu(df)\\
	& = \sum_{l=1}^k c_l a(f_l)
\end{align*}
Finally, when $k < n,$ there is a unique extension of $x$ to an infinite positive definite sequence and only a unique measure $\mu$ with $x$ for its moments.
\end{theorem}

This theorem can be proved using Herglotz theorem~\cite{herglotz} and the
theorems on flat extensions of moment sequences studied in~\cite{Curto97}. We
refer the interested reader to~\cite{grenander01} for an algebraic proof of this
theorem. A straightforward corollary of Caratheodory's theorem is the following
Vandermonde Decomposition for positive definite Toeplitz matrices.

\begin{corollary}[Vandermonde Decomposition]\label{lm:vand} Any positive semidefinite Toeplitz matrix $P \in \C^{n \times n}$ can be represented as
  follows
  \[
    P  =  V D V^*,
  \]
  where
  \[
  \begin{aligned}
    V & = \left[ {a} \left( f_1 \right) \cdots
    {a} \left( f_r \right) \right]\,,\\
    D & = \operatorname{diag} \left( \left[ d_1 \cdots d_r \right] \right)\,,
  \end{aligned}
  \]
 $d_k$ are real positive numbers, and $r = \rank(P)$.
\end{corollary}
\begin{proof}
Write $P$ as $T_n(x)$ for some $x \in \C^n$. By assumption, $T_n(x) \succeq 0$ and $\rank(T_n(x)) = r $. Therefore by Caratheodory-Toeplitz theorem, $x$ can be written as $\sum_{l=1}^r{d_l a(f_l)}$. Thus, 
\begin{align*}
	P = T_n(x) &= \sum_{l=1}^r{d_l T_n(a(f_l))}\\
	&=\sum_{l=1}^r{d_l a(f_l) a(f_l)^*}\\
	& = V D V^*,
\end{align*}
where $V$ and $D$ are defined as in the statement of the theorem.
\end{proof}

% section preliminaries (end)

% \section{Duality : Moments and Polynomials}
% The sets $\cone(\A_+)$ (resp $\cone(\A$)) and $\cone(\A^*_+)$ (resp
% $\cone(\A^*$)) are dual cones, a fact that can be easily checked with a little
% computation. Interestingly, for the positive case, the characterizations of the
% conical hulls of the atomic norm balls are sufficient to characterize the norm
% balls themselves. The general case requires a little more computation.
% 
% \section{Positive Trigonometric Moments}
% 
% The conical hull of the atomic set $\A_+$ given by $\cone(\A_+)$ is called the
% trigonometric moment cone and has a classical characterization due to
% Caratheodory-Toeplitz theorem.

\section{SDP for Trigonometric Moments} % (fold)
\label{sec:sdp_for_trigonometric_moments}


\subsection{Positive Trigonometric Moments} % (fold)
\label{sub:positive_trigonometric_moments}

As a consequence of the Caratheodory's characterization of the moment cone , we can easily prove the  characterizion of the the convex hull of $\A_+$, given by Theorem~\ref{thm:positive-linespect-sdp}.

If $T x \succeq 0$, then there exists an atomic decomposition $x = \sum_l c_l a(f_l)$ by Caratheodory-Toeplitz theorem since it is in the conical hull of the atomic set $\A_+$. Since we have an invariant $x_0 = \sum_l c_l$ for any such decomposition, we have that
\[
\vnorm{x}_{\A_{+}}  = \inf\setof{\sum_l{c_l}}{x = \sum_l c_l} = x_0.
\]
On the other hand, if $Tx \not\succeq 0$, $x \not\in \cone(\A_+)$ and thus $\vnorm{x}_{\A_{+}} = +\infty$.


% subsection positive_trigonometric_moments (end)

\subsection{General Trigonometric Moments} % (fold)
\label{sub:general_trigonometric_moments}

For the general trigonometric atoms, let us denote the atoms by two indices for  frequency and phase,
\[
	a(f,\phi) = a(f) \exp(i 2\pi \phi)
\]
so that $\A = \setof{a(f,\phi)}{f, \phi \in \mathbb{T}}.$

\begin{proof}[Proof of Theorem~\ref{thm:sdp-char}]
Denote the value of the right hand side of \eq{atomic_sdp} by $\mathrm{SDP}(x)$.  Suppose $x = \sum_k c_k a(f_k,\phi_k)$ with $c_k>0$.  Defining $u = \sum_k c_k a(f_k,0)$ and $t = \sum_k c_k$, we note that 
\[
T(u) = \sum_{k}c_k a(f_k,0) a(f_k,0)^* = \sum_{k}c_k a(f_k,\phi_k) a(f_k,\phi_k)^*.
\] 
Therefore, 
\begin{align}\label{eqn:verifyfeasibility}
		\begin{bmatrix}
T (u) & x\\
x^* & t
\end{bmatrix}
= \sum_{k} c_k \begin{bmatrix} a(f_k,\phi_k) \\ 1 \end{bmatrix}\begin{bmatrix} a(f_k,\phi_k) \\ 1 \end{bmatrix}^* 		 \succeq 0
	\end{align}

Now, $\frac{1}{n}\trace(T(u)) = t = \sum_k c_k$ so that $SDP(x) \leq \sum_k c_k$. Since this holds for any decomposition of $x$, we conclude that $\|x\|_{\mathcal{A} }\geq \mathrm{SDP}(x)$.

Conversely,  suppose for some $u$ and $x$,
\begin{equation}\label{eq:toep-psd}
\begin{bmatrix}
T (u) & x\\
x^* & t
\end{bmatrix} \succeq 0\,.
\end{equation}
In particular, $T (u)\succeq 0$.  Form a Vandermonde decomposition
\[
	T(u)= V D V^*
\]
as promised by Lemma~\ref{lm:vand}. Since $V D V^* = \sum_k d_k a(f_k,0)
a(f_k,0)^*$ and $\|a(f_k,0)\|_2=\sqrt{n}$, we have $\frac{1}{n}\tr(T (u)) =
\tr(D)$.

Using this Vandermonde decomposition and the matrix inequality~\eq{toep-psd}, it
follows that $x$ is in the range of $V$, and hence
\[
	x = \sum_k w_k a(f_k,0) = Vw
\]
for some complex coefficient vector $w = [\cdots, w_k, \cdots]^T$.  Finally, by the Schur Complement Lemma, we have
\[
	V D V^* \succeq t^{-1} V w w^* V^*
\]
Let $q$ be any vector such that $V^*q = \operatorname{sign}(w)$.  Such a vector exists because $V$ is full rank.  Then
\[
	\tr(D)= q^* V D V^*q \succeq t^{-1}q^* V w w^* V^*q = t^{-1} \left(\sum_k |w_k|\right)^2.
\]
implying that $\tr(D) t \geq \left(\sum_k |w_k|\right)^2$.   By the arithmetic geometric mean inequality,
\[
	\tfrac{1}{2n} \tr(T (u)) + \tfrac{1}{2} t = 	\tfrac{1}{2} \tr(D) + \tfrac{1}{2} t \geq \sqrt{\tr(D) t} \geq \sum_k |w_k| \geq \|x\|_\A
\]
implying that $\mathrm{SDP}(x)\geq \|x\|_{\mathcal{A}}$ since the previous chain of inequalities hold for any choice of $u,t$ that are feasible.
\end{proof}
% subsection general_trigonometric_moments (end)

% section sdp_for_trigonometric_moments (end)

\section{SDP for Trigonometric Polynomials} % (fold)
\label{sec:sdp_for_trigonometric_polynomials}

The dual problem to AST involves trigonometric polynomials instead of moments.
\subsection{Positive Trigonometric Polynomials}

\begin{definition}
A vector $q \in \C^n$ is a positive trigonometric polynomial if
for every $f \in \mathbb{T},~ \Re\sum_{j=1}^{n} q_{j-1} \exp(i2\pi j f) \geq 0.$
\end{definition}

Such polynomials have a simple characterization due to spectral
factorization theorem.

\begin{theorem}\label{thm:gram-mtx-positive-poly}
$q \in \C^n$ is a positive polynomial if and only if there exists $Q \succeq 0$ such that $T^*Q = q.$
\end{theorem}

\subsection{General Trigonometric Polynomials}

Recall from \eqref{eq:dual-norm-poly} that the dual atomic norm of a vector $v
\in \mathbb{C}^n$ is the maximum absolute value of a complex trigonometric
polynomial $V(f) = \sum_{l=0}^{n-1} v_l e^{-2\pi i l f}$. As a
consequence, a constraint on the dual atomic norm is equivalent to
a bound on the magnitude of $V(f)$:
\begin{align*}
\|v\|_\A^* \leq \tau \Leftrightarrow |V(f)|^2 \leq \tau^2, \forall f \in [0, 1].
\end{align*}
The function $q(f) = \tau^2-|V(f)|^2$ is a trigonometric polynomial (that is, a
polynomial in the variables $z$ and $z^*$ with $|z|=1$). A necessary and
sufficient condition for $q(f)$ to be nonnegative is that it can be written as
a sum of squares of trigonometric polynomials~\cite{Megretski03}. 
Testing if $q$ is a sum of squares can be achieved
via semidefinite programming. Let $T^*$ denote the adjoint of the map $T$. Then we have the following
succinct characterization

\begin{lemma}\cite[Theorem 4.24]{brl2007}\label{lm:brl} For any given causal trigonometric polynomial $V(f) = \sum_{l=0}^{n-1} v_l
e^{-2\pi i l f}$, $|V(f)| \leq \tau $ if and only if there exists complex
Hermitian matrix $Q$ such that
\begin{align*}
T^*(Q) = \tau^2 {e}_1~~\mbox{and}~~
\begin{bmatrix}
  Q & v \\
  v^* & 1
 \end{bmatrix} \succeq 0.
\end{align*}
Here, ${e}_1$ is the first canonical basis vector with a one at the first
component and zeros elsewhere and $v^*$ denotes the Hermitian adjoint
(conjugate transpose) of $v$.
\end{lemma}

\subsection{Deriving the primal characterization from dual}
\label{sec:sdp-ast}

In this section, we present a semidefinite characterization of the atomic norm
associated with the line spectral atomic set $\mathcal{A} = \{a_{f,\phi} | f
\in [0, 1], \phi \in [0, 1]\}$. This characterization allows us to rewrite
 {\eqref{AST}} as an equivalent semidefinite programming problem.



Using Lemma \ref{lm:brl}, we rewrite the atomic norm $\|x\|_\A =
\sup_{\|v\|_\A^*\leq 1} \left<x, v\right> $ as the following semidefinite
program:
\begin{equation}\label{eq:sdpprimal}
 \begin{array}{ll}
\operatorname*{maximize}_{v,\ Q} & \left<x, v\right>\\
\text{subject  to}  &
 \begin{bmatrix}
  Q & v \\
  v^* & 1
 \end{bmatrix} \succeq 0, \qquad  T^*(Q) = {e}_1 .
\end{array}
\end{equation}
The dual problem of \eq{sdpprimal} (after a trivial rescaling) is then equal to
the atomic norm of $x$:
\begin{align*}
\begin{array}{lll} \|x\|_\A =&  \min_{t, u} & \tfrac{1}{2} (t + u_1)  \\
&\operatorname{subject\ to}
& \begin{bmatrix}
  T(u) & x \\
  x^* & t
 \end{bmatrix} \succeq 0.\end{array}
\end{align*}
Therefore, the atomic denoising problem \eqref{AST} for the set of trigonometric atoms is equivalent to
\begin{equation}\label{eq:sdpdenoising}
\begin{array}{ll}
\operatorname*{minimize}_{t, u, x} & \frac{1}{2} \|x - y\|_2^2 + \frac{\tau}{2}(t + u_1) \\
\operatorname{subject\ to}
& \begin{bmatrix}
  T(u) &  x \\
 x^* & t
 \end{bmatrix} \succeq 0.\end{array} 
\end{equation}

The semidefinite program \eq{sdpdenoising} can be solved by off-the-shelf
solvers such as SeDuMi~\cite{sedumi} and SDPT3~\cite{SDPT3}. However, these
solvers tend to be slow for large problems. In the next section, we provide a
reasonably efficient algorithm based upon the Alternating Direction Method of
Multipliers.

% section sdp_for_trigonometric_polynomials (end)

\section{Alternating Direction Method of Multipliers}
\label{sec:admm}

A thorough survey of the ADMM algorithm is given in~\cite{admm2011}. We only
present the details essential to the implementation of atomic norm soft
thresholding. To put our problem in an appropriate form for ADMM,
rewrite~\eq{sdpdenoising} as
\begin{equation*}
\begin{array}{ll}
\operatorname*{minimize}_{t, u, x,Z} & \frac{1}{2} \|x - y\|_2^2 + \frac{\tau}{2}(t + u_1) \\
\operatorname{subject\ to}
& Z=\begin{bmatrix}
  T(u) & x \\
  x^* & t
 \end{bmatrix} \\
& Z\succeq 0.\end{array} 
\end{equation*}
and dualize the equality constraint via an Augmented Lagrangian:
\begin{align*}
\mathcal{L}_\rho (t,u,x,Z, \Lambda)= \frac{1}{2} \|x - y\|_2^2 + \frac{\tau}{2}(t +  u_1) + \\
\left\langle   \Lambda, Z-\begin{bmatrix}
  T(u) & x \\
  x^* & t
 \end{bmatrix} \right\rangle +
  \frac{\rho}{2} \left\| Z-\begin{bmatrix}
  T(u) & x \\
  x^* & t
 \end{bmatrix} \right\|_F^2
\end{align*}

ADMM then consists of the update steps:
\begin{align*}
(t^{l+1},u^{l+1},x^{l+1})  \leftarrow \arg\min_{t,u,x} \mathcal{L}_\rho(t,u,x,Z^l, \Lambda^l) \\
Z^{l+1}  \leftarrow \arg\min_{Z\succeq 0}  \mathcal{L}_\rho(t^{l+1},u^{l+1},x^{l+1}, Z, \Lambda^l ) \\
\Lambda^{l+1}  \leftarrow \Lambda^l + \rho  \left( Z^{l+1}-\begin{bmatrix}
  T(u^{l+1}) &  x^{l+1} \\
  {x^{l+1}}^* & t^{l+1}
 \end{bmatrix} \right).
\end{align*}
The updates with respect to $t$, $x$, and $u$ can be computed in closed form:
\begin{align*}
    t^{l+1} &= Z_{n+1,n+1}^l+\left(\Lambda_{n+1,n+1}^l-\frac{\tau}{2}\right)/\rho\\
	 x^{l+1} &= \frac{1}{2\rho+1}(y  + 2\rho z_1^{l} + 2\lambda_1^l)\\
    u^{l+1} &= W \left(T^*(Z_0^l+ \Lambda_0^l/\rho) - \frac{\tau }{2\rho} {e}_1\right)
\end{align*}
Here $W$ is the diagonal matrix with entries
\[
	W_{ii} = \begin{cases} 
		\frac{1}{n} & i=1\\
		\frac{1}{2(n-i+1)} & i>1
	\end{cases}
\]
and we introduced the partitions:
\[
	Z^l = \begin{bmatrix} Z_0^l & z_1^l \\ {z_1^l}^* & Z^l_{n+1,n+1} \end{bmatrix} ~~~\mbox{and}~~~
	\Lambda^l = \begin{bmatrix} \Lambda_0^l & \lambda_1^l \\ {\lambda_1^l}^* & \Lambda^l_{n+1,n+1} \end{bmatrix}\,.
\]
The $Z$ update is simply the projection onto the positive definite cone
\begin{equation}\label{eq:z-step}
\hspace{-.3cm}Z^{l+1}:=	 \arg\min_{Z\succeq 0} \left\|Z-\begin{bmatrix}
  T(u^{l+1}) &  x^{l+1} \\
   {x^{l+1}}^* & t^{l+1}
 \end{bmatrix}+\Lambda^{l}/\rho\right\|_F^2\,.
\end{equation}
Projecting a matrix $Q$ onto the positive definite cone is accomplished by
forming an eigenvalue decomposition of $Q$ and setting all negative eigenvalues
to zero.

To summarize, the update for $(t,u,x)$ requires averaging the diagonals of a
matrix (which is equivalent to projecting a matrix onto the space of Toeplitz
matrices), and hence operations that are $O(n)$. The update for $Z$ requires
projecting onto the positive definite cone and requires $O(n^3)$ operations. The
update for $\Lambda$ is simply addition of symmetric matrices.

Note that the dual solution $\hat{z}$ can be obtained as $\hat{z} = y - \hat{x}$
from the primal solution $\hat{x}$ obtained from ADMM by using
Lemma~\ref{lem:dual-problem}.

\section{Discretization}

In general the atomic sets may not have a computable characterization. Even if
there is an SDP characterization and an efficient implementation using
techniques like ADMM, when the number of samples is larger than a few hundred,
the running time of our ADMM method is dominated by the cost of computing
eigenvalues and is usually expensive. This is not avoidable as there is no cheap
way to find projections on the positive definite cone. For very large problems,
we now propose using discretization and hence Lasso as an alternative to the
semidefinite program~\eq{sdpdenoising}.

\subsection{Discretized Atomic Soft Thresholding} % (fold)
\label{sub:discretized_atomic_soft_thresholding}

Suppose we solve AST~\eqref{AST} on a different set $\widetilde{\A}$ (say, an
$\epsilon$-net of $\A$) instead of $\A$. If for some $M>0,$ \[
M^{-1}\vnorm{x}_{\widetilde{\A}} \leq \vnorm{x}_{\A} \leq
\vnorm{x}_{\widetilde{\A}} \] holds for every $x$, then Theorem
$\ref{cor:expected-mse}$ still applies with a constant factor $M$. Our
justification for using the finite dimensional Lasso as an alternative to the
general infinite atomic norm soft thresholding problem relies on approximation
guarantees for epsilon-nets of the atomic sets. To be precise, in the next
section we see that $M$ approaches unity as $\epsilon \to 0.$ Thus, the solution
to the discretized atomic soft thresholding (DAST) problem approaches the AST
solution.

The following proposition demonstrates that the universal guarantee in Theorem
~\ref{cor:expected-mse} continues to hold with only a penalty of a small
multiplicative constant when DAST is used in place of AST.
\begin{prop}\label{prop:grid-approx-mse}
Suppose 
\begin{equation}
  \label{dni}
  \vnorm{z}_{\widetilde{\A}}^* \leq \vnorm{z}_{\A}^* \leq 
  M\vnorm{z}_{\widetilde{\A}}^* \text{ for every } z,
\end{equation}
or equivalently 
\begin{equation}
  \label{pni}
  M^{-1}\vnorm{x}_{\widetilde{\A}} \leq \vnorm{x}_{\A} \leq \vnorm{x}_{\widetilde{\A}} \text{ for every } x,
\end{equation}
then under the same conditions as in Theorem \ref{cor:expected-mse},
\begin{equation*}
    \frac{1}{n} \E \vnorm{\tilde{x} - x^\star}_2^2 \leq \frac{M \tau}{n}\vnorm{x^\star}_\A
\end{equation*}
where $\tilde{x}$ is the optimal solution for \eqref{AST} with $\A$ replaced by $\widetilde{\A}.$
\end{prop}
\begin{proof}\belowdisplayskip=-12pt
By assumption, $\E\left(\vnorm{w}_{\A}^*\right) \leq \tau$. Now, \eqref{dni}
implies $\E\left(\vnorm{w}_{\widetilde{\A}}^*\right) \leq \tau.$ Applying
Theorem \ref{cor:expected-mse}, and using \eqref{pni}, we get
\begin{equation*}
\frac{1}{n} \E \vnorm{\tilde{x} - x^\star}_2^2 \leq \frac{\tau}{n}\vnorm{x^\star}_{\widetilde{\A}} \leq \frac{M \tau}{n}\vnorm{x^\star}_{\A}.
\end{equation*}
\end{proof}

% Using the results for approximated atomic norms, we can suggest the following
% alternative to Atomic soft thresholding~\eqref{AST} and instead solve AST on a
% uniform grid of $N$ atoms, where $N$ is the size of the $\epsilon$-net. We call
% this Discretized Atomic Soft Thresholding or DAST. In the following, we will use
% the notation $\A_N$ in place of $\A_\epsilon$ in order to explicitly
% characterize the quality of approximation in terms of the grid size.
% 
% As the grid size increases, the solution to DAST converges to AST. In this section, we will show that
% TODO: Would be nice to characterize AST convergence rate.
% Now, we will turn our attention to how well the AST solution is approximated
% when we replace $\A$ with $\A_N$. This can be characterized in terms of the
% approximation of atomic norms using the optimality conditions of AST. Let
% $\hat{x}_N$ and $\hat{x}$ be the solution to DAST and AST respectively. Then,
% \begin{align}
% \vnorm{\hat{x}_N - \hat{x}}_2^2 &=  \vabs{\hat{x}_N - \hat{x}}{(y - \hat{x}) - (y - \hat{x}_N)}\\
% &=  \tau \vabs{\hat{x}_N - \hat{x}}{\hat{q} - \hat{q}_N}\\
% &=  \tau \vabs{\hat{x}_N - \hat{x}}{\hat{q} - \hat{q}_N}
% \end{align}

% subsection discretized_atomic_soft_thresholding (end)


\subsection{Approximated Atomic Norms} % (fold)
\label{sub:approximated_atomic_norms}

In this section, we will show that atomic norms can be approximated by choosing
a large $\epsilon$-net of atoms instead of the original set of atoms. Our proof
of approximation and characterization of the convergence depends upon using an
Euclidean $\epsilon$ cover of the atomic set $\A$, and the equivalence between
Euclidean and atomic norms in finite dimensions.

If $\A \in F^n$, there exists $C$, possibly depending on $n$ such that
$\vnorm{x}_\A \leq C \vnorm{x}_2$ for any choice of atomic set. When the atomic
set is symmetric, it is often possible to find the optimal choice of $C$ by
finding the minimum volume ellipsoid that circumscribes $\conv(\A)$, in the
special case that the ellipsoid is a scaled version of the Euclidean ball. This
is called the L\"{o}wner-John ellipsoid after Charles L\"{o}wner who discovered
the uniqueness of the minimum volume circumscribing ellipsoid for any convex
sets and Fritz John who proved the existence.

Using the characterization of John, we have the following condition for
Euclidean ball to be the minimum volume circumscribing ellipsoid of $\conv(\A).$

\begin{theorem}\label{thm:john:ellipsoid}
If there exists positive numbers $\lambda_1, \dots, \lambda_m >0$ and atoms
$a_1, \dots, a_m$ for $m \geq n$ such that
\[
		\sum_{i=1}^m \lambda_i a_i = 0 \text{ and } I_n = \sum_{i=1}^m \lambda_i (a_i a_i^*),
\]
the Euclidean ball $B=\setof{x}{\vnorm{x}_2 \leq 1}$ is the unique minimum
volume ellipsoid that circumscribes $\conv(\A)$. If in addition, $\A$ is
centrosymmetric, for any $x$,
\[
	\vnorm{x}_2 \leq \vnorm{x}_\A  \leq \sqrt{n} \vnorm{x}_2 
\]	
\end{theorem}

The conditions in the previous theorem hold for a number of interesting atomic
set including Fourier measurements discussed in Chapter~\ref{chap:linespect},
and one may argue that the minimum volume ellipsoid is often a Euclidean sphere.
When the condition in Theorem~\ref{thm:john:ellipsoid} is satisfied, defining
$\A_\epsilon$ as an $\epsilon$-net of $\A$, we can write
\begin{align*}
	\vnorm{z}_\A^* &= \sup_{x \in \A} \vabs{z}{x}\\
	&\leq \sup_{\hat{x} \in \A_\epsilon} \vabs{z}{\hat{x}} + \inf_{\hat{x} \in \A_\epsilon} \vnorm{z}_\A^* \vnorm{x - \hat{x}}_\A\\
	&\leq \sup_{\hat{x} \in \A_\epsilon} \vabs{z}{\hat{x}} +  \vnorm{z}_\A^* \sup_{\vnorm{w}_2 \leq \epsilon}{w}_\A\\
	&\leq \vnorm{z}_{\A_\epsilon}^* +  \sqrt{n}\epsilon\vnorm{z}_\A^*,
\end{align*}
whence we get $\vnorm{z}_\A^* \leq
(1-\sqrt{n}\epsilon)^{-1}\vnorm{z}_{\A_\epsilon}^*$. Also, since $\A_\epsilon
\subset \A,$ by definition $\vnorm{z}_{\A_\epsilon}^* \leq \vnorm{z}_\A^*$.
Combining these two, we have the following approximation
\begin{equation}
\label{eq:enet-approx-dual}	
\vnorm{z}_{\A_\epsilon}^* \leq \vnorm{z}_\A^* \leq (1-\sqrt{n}\epsilon)^{-1}\vnorm{z}_{\A_\epsilon}^*.
\end{equation}

We will need the following lemma to restate the $\epsilon$ approximation in
terms of the atomic norm, instead of dual.

\begin{lemma}
${\vnorm{z}_{\A}^* \leq M\vnorm{z}_{\widetilde{\A}}^*}$ for every $z$ iff
${M^{-1}\vnorm{x}_{\widetilde{\A}} \leq \vnorm{x}_{\A}}$ for every $z$.
\end{lemma}
\begin{proof}\belowdisplayskip=-12pt
We will show the forward implication -- the converse will follow since the dual
of the dual norm is again the primal norm. By \eqref{holder}, for any $x$, 
there exists a $z$ with
$\vnorm{z}_{\widetilde{\A}}^* \leq 1$ and ${\vabs{x}{z} =
\vnorm{x}_{\widetilde{\A}}}$. So,
\begin{align*}
M^{-1}\vnorm{x}_{\widetilde{\A}} &= M^{-1}\vabs{x}{z} &&\\
&\leq M^{-1}\vnorm{z}_{\A}^* \vnorm{x}_{\A} &&\text{by \eqref{holder}}\\
&\leq \vnorm{x}_{\A} &&\text{by assumption.}
\end{align*}
\end{proof}

Now, we can write the approximation for atomic norm. For any $x,$
\begin{equation}
\label{eq:enet-approx}	
(1-\sqrt{n}\epsilon)\vnorm{x}_{\A_\epsilon} \leq \vnorm{x}_\A \leq \vnorm{x}_{\A_\epsilon}.
\end{equation}

While this method is generic, simpler and sometimes stronger arguments may be possible for specific atomic sets. In succeeding sections, we use techniques similar to what we just outlined and characterize the atomic norm approximation for grids of the atomic set.

% subsection approximated_atomic_norms (end)

\subsection{DAST for Line Spectral Signals}\label{sec:comp-method}

To proceed, pick a uniform grid of $N$ frequencies and form $\A_N = \left\{
a_{m/N,\phi} ~\middle|~ 0 \leq m \leq N-1 \right\} \subset \A $ and solve
\eqref{AST} on this grid. i.e., we solve the problem
\begin{equation}
	\label{epsprimal} \text{minimize }\frac{1}{2} \vnorm{x - y}_2^2 + \tau \vnorm{x}_{\A_N}. 
\end{equation}

To see why this is to our advantage, define $\Phi$ be the $n \times N$ Fourier
matrix with $m$th column $a_{m/N,0}$. Then for any $x \in \C^n$ we have $\vnorm{x}_{\A_N} = \min\left\{ \vnorm{c}_1: x = \Phi c \right\}$.
So, we solve
\begin{equation}
	\label{sparsa} \text{minimize }\frac{1}{2} \vnorm{\Phi c- y}_2^2 + \tau \vnorm{c}_1. 
\end{equation}
for the optimal point $\hat{c}$ and set $\hat{x}_N = \Phi \hat{c}$ or the first
$n$ terms of the $N$ term discrete Fourier transform (DFT) of $\hat{c}$.
Furthermore, $\Phi^* z$ is simply the $N$ term inverse DFT of $z \in \C^n$.
This observation coupled with Fast Fourier Transform (FFT) algorithm for
efficiently computing DFTs gives a fast method to solve \eqref{epsprimal},
using standard compressed sensing software for $\ell_2-\ell_1$ minimization,
for example, SparSA~\cite{wright09}.

Because of the relatively simple structure of the atomic set, the optimal
solution $\hat{x}$ for \eqref{epsprimal} can be made arbitrarily close to
\eqref{eq:sdpdenoising} by picking $N$ a constant factor larger than $n$. In
fact, the following section furnishes the proof that the atomic norms induced by $\A$ and a discretized version $\A_N$ are equivalent.

Due to the efficiency of the FFT, the discretized approach has a much lower
algorithmic complexity than either Cadzow's alternating projections method or
the ADMM method described in the extended technical report~\cite{btr12}, which
each require computing an eigenvalue decomposition at each iteration. Indeed,
fast solvers for~\eqref{sparsa} converge to an $\epsilon$ optimal solution in no
more than $1/\sqrt{\epsilon}$ iterations. Each iteration requires a
multiplication by $\Phi$ and a simple ``shrinkage'' step. Multiplication by
$\Phi$ or $\Phi^*$ requires $O(N\log N)$ time and the shrinkage operation can be
performed in time $O(N)$.

As we discuss below, this fast form of basis pursuit has been proposed by
several authors. However, analyzing this method with tools from compressed
sensing has proven daunting because the matrix $\Phi$ is nowhere near a
restricted isometry. Indeed, as $N$ tends to infinity, the columns become more
and more coherent. However, common sense says that a larger grid should give
better performance, for both denoising and frequency localization! Indeed, by
appealing to the atomic norm framework, we are able to show exactly this point:
the larger one makes $N$, the closer one approximates the desired atomic norm
soft thresholding problem. Moreover, we do not have to choose $N$ to be too
large in order to achieve nearly the same performance as the AST.

\subsubsection{Approximation of the Dual Atomic Norm}
\label{proof:dual-norm-approximation}
Note that the dual atomic norm of $w$ is given by
\begin{equation}
  \label{eq:maximum-modulus}
  \vnorm{w}_\A^* = \sqrt{n}\sup_{f \in [0,1]} \left| W_n(e^{i2\pi f}) \right|.
\end{equation}
i.e., the maximum modulus of the polynomial $W_n$ defined by
\begin{equation}
\label{eq:random-poly}
W_n(e^{i2\pi f}) =\frac{1}{\sqrt{n}} \sum_{m=0}^{n-1}{w_m e^{-i 2 \pi m f}}.
\end{equation}
Treating $W_n$ as a function of $f$,  with a slight abuse of  notation, define
\begin{equation*}
\vnorm{W_n}_\infty := \sup_{f \in [0,1]} \left| W_n(e^{i2\pi f}) \right|.
\end{equation*}
We show that we can approximate the maximum modulus by evaluating $W_n$ in a
uniform grid of $N$ points on the unit circle. To show that as $N$ becomes large,
the approximation is close to the true value, we bound the derivative of $W_n$ 
using Bernstein's inequality for polynomials.

\begin{theorem}[Bernstein, See, for example ~\cite{schaeffer41}]
Let $p_n$ be any polynomial of degree $n$ with complex coefficients. Then,
\begin{equation*}
 \sup_{|z|\leq 1} |p'(z)|  \leq n  \sup_{|z|\leq 1} |p(z)|.
\end{equation*}
\end{theorem}
Note that for any $f_1, f_2 \in [0,1],$ we have
\begin{align*}
  \left|W_n(e^{i2\pi f_1})\right| - \left|W_n(e^{i2\pi f_2})\right|  &\leq   \left|e^{i2\pi f_1} - e^{i2\pi f_2}\right| \vnorm{W_n'}_\infty&&\\
  &= 2|\sin(2\pi (f_1-f_2))| \vnorm{W_n'}_\infty\\
  &\leq 4 \pi (f_1-f_2) \vnorm{W_n'}_\infty\label{diff}\\
  & \leq 4 \pi n (f_1-f_2) \vnorm{W_n}_\infty,
\end{align*}
where the last inequality follows by Bernstein's theorem.
Letting $s$ take any of the $N$ values $0,\tfrac{1}{N} \ldots, \tfrac{N-1}{N}$, we see that,
\begin{equation*}
\vnorm{W_n}_\infty \leq \max_{m=0,\ldots,N-1}\left| W_n\left(e^{i 2 \pi m/N}\right) \right| + \frac{2\pi n}{N}\vnorm{W_n}_\infty.
\end{equation*}
Since the maximum on the grid is a lower bound for maximum modulus of $W_n$, we have
\begin{align}
\max_{m=0,\ldots,N-1} \left| W_n\left(e^{i 2 \pi m/N}\right) \right| \leq \vnorm{W_n}_\infty \\
\leq  \left(1-\frac{2\pi n}{N}\right)^{-1} \max_{m=0,\ldots,N-1} \left| W_n\left(e^{i 2 \pi m/N}\right) \right|\nonumber\\
 \leq  \left(1+\frac{4\pi n}{N}\right) \max_{m=0,\ldots,N-1} \left| W_n\left(e^{i 2 \pi m/N}\right) \right|.
\label{eq:grid-approx}
\end{align}
Thus, for every $w,$
\begin{equation}
\vnorm{w}_{\A_N}^* \leq  \vnorm{w}_\A^* \leq  \left(1-\frac{2\pi n}{N}\right)^{-1} \vnorm{w}_{\A_N}^*
\end{equation}
or equivalently, for every $x,$
\begin{equation}
 \left(1-\frac{2\pi n}{N}\right) \vnorm{x}_{\A_N} \leq  \vnorm{x}_\A \leq \vnorm{x}_{\A_N}
\end{equation}

\begin{equation}
 \left(1-\frac{2\pi n}{N}\right) \vnorm{x}_{\A_N} \leq  \vnorm{x}_\A \leq \vnorm{x}_{\A_N}, \forall x \in \C^n
\end{equation}
Using Proposition
\ref{prop:grid-approx-mse} and \eq{tau}, we conclude
{\small
\begin{align*}
\frac{1}{n} \E \vnorm{\hat{x}_N - x^\star}_2^2 
&\leq
\frac{\sigma\left(\frac{\log(n)+1}{\log(n)}\right)\vnorm{x^\star}_\A
\sqrt{ n \log(n) + 
    n\log(4\pi\log(n))
}}{n\left(1 - \frac{2\pi n}{N}\right)} =
O\left(
\sigma \sqrt{\frac{\log(n)}{n}} \frac{ \vnorm{x^\star}_\A}{\left(1 - \frac{2\pi n}{N}\right)}
\right)
\end{align*}
}


\subsection{DAST for System Identification} % (fold)
\label{sec:dast_for_system_identification}
The following proposition asserts that we can approximate this finite dimensional atomic norm via a sufficiently fine discretization of the unit disk.

\begin{prop}\label{prop:algos:grid}
Let $\D_\rho^{(\epsilon)}$ be a finite subset of the unit disc such that for any $w \in \D_\rho$ there exists a $v \in \D_\rho^{(\epsilon)}$ satisfying $|w-v| \leq \epsilon$.  Define
\[
	\|x\|_{\cL(\cA_\epsilon)} = \inf\left\{ \sum_{w\in \D_{\rho}^{(\epsilon)}} |c_w| ~:~ x_i = \sum_{w\in\D_{\rho}^{(\epsilon)}} c_w\mathcal{L}_i\left(\frac{1-|w|^2}{z-w} \right)\right\}\,.
\]
Then there exists a constant $C_\epsilon \in [0,1]$ such that
\[
	C_\epsilon \|x\|_{\cL(\cA_{\epsilon})} \leq \|x\|_{\cL(\cA)} \leq \|x\|_{\cL(\cA_{\epsilon})} \,.
\]
\end{prop}
\noindent The set $\D_\rho^{(\epsilon)}$ is called an $\epsilon$-net for the set $\D_\rho$.  We show in the appendix that when $\mathcal{L}_k(G) = G(e^{i\theta_k})$, $C_{\epsilon}$ is at least $(1-\tfrac{16 \rho \epsilon}{\pi(1-\rho)})$.  Other measurement ensembles can be treated similarly.

When we replace $\|x\|_{\cL(\cA)}$ with its discretized counterpart $\|x\|_{\cL(\cA)}$ in~\eq{finite-dim-ast}, 
\[
	\minimize_{x} \tfrac{1}{2}\|x-y\|_2^2 + \mu \|x\|_{\cL(\cA_\epsilon)}
\]
is equivalent to
\begin{equation}\label{eq:algos:DAST}
	\minimize_{c}  \tfrac{1}{2}\|Mc-y\|_2^2 + \mu \sum_{w\in \D_\rho^{(\epsilon)}} |c_w|
\end{equation}
where
\[
	M_{ij} = \mathcal{L}_i \left(\tfrac{1-|w_j|^2}{z-w_j}\right)
\]
and $j$ indexes the set $\D_{\rho}^{(\epsilon)}$. That is $M$ is an $n \times
|\D_\rho^{(\epsilon)}|$ matrix. Problem~\eq{algos:DAST} is a weighted $\ell_1$
regularization problem with real or complex data depending on specific problem.
We call~\eq{algos:DAST} \emph{Discretized Atomic Soft Thresholding} (DAST), as coined in~\cite{btr12}.

The DAST problem can be solved very efficiently with a variety of off-the-shelf
tools including SPARSA~\cite{wright09}, FPC~\cite{Hale08} or even more general
purpose packages such as YALMIP~\cite{YALMIP} or CVX~\cite{cvx}. DAST yields an
approximate solution to problem~\eq{atomic-norm-lin-inverse}, and, as we will
see, yields a statistically consistent estimate provided the parameter
$\epsilon$ is adjusted to meet the desired numerical accuracy.

\begin{proof}\label{pf:prop-grid}
First note that for any atomic sets $\cA \subset \cA'$, $\|x\|_{\cA'} \leq \|x\|_{\cA}$.  The harder part of this proposition is the lower bound.  To proceed, we use the dual norm. 

Let $\D_\rho^{(\tau)}$ be a subset of $\D_\rho$ such that for every $a \in \D_\rho$, there exists an $\hat{a}\in \D_\rho^{(\tau)}$ satisfying $|a-\hat{a}|\leq \tau$.  For each $a \in \D_\rho$, we will actually denote $\hat{a}$ as the closest point in $\D_\rho^{(\tau)}$ to $a$.

Now observe that 
\begin{align*}
	  \| \cL(\varphi_{\hat{a}}-\varphi_a)\|_{\cL(\cA)} \leq  \| \varphi_{\hat{a}}-\varphi_a\|_{\cA}
	  \leq \frac{8}{\pi} \|\Gamma_{\varphi_{\hat{a}}}- \Gamma_{\varphi_a}\|_1 \leq \frac{16 \rho \tau}{\pi (1-\rho)}\,.
\end{align*}
Here, the first inequality follows from our reasoning in Section~\ref{sec:computation}.  The second inequality is Theorem~\ref{thm:hankel-nuclearity}, and the final inequality is by~\eq{hankel-op-bound}.

We then can compute
\begin{align*}
\|z\|_{\cL(\cA)}^* &= \sup_{a \in \D_\rho} \langle \cL(\varphi_a),z \rangle\\
&= \sup_{a \in \D_\rho} \langle \cL(\varphi_{\hat{a}}), z \rangle + \langle \cL(\varphi_{a}-\varphi_{\hat{a}}), z\rangle \\
&\leq\sup_{a \in \D_\rho^{(\tau)}} \langle \cL(\varphi_{\hat{a}}), z \rangle +\sup_{a \in \D_\rho}\langle \cL(\varphi_a-\varphi_{\hat{a}}), z\rangle \\
&= \|z\|_{\cL(\cA_\tau)}^*+ \sup_{a \in \D_\rho}\langle \cL(\varphi_a-\varphi_{\hat{a}}), z\rangle \\
&\leq \|z\|_{\cL(\cA_\tau)}^*+ \sup_{a \in \D_\rho}  \| \cL(\varphi_{\hat{a}}-\varphi_a)\|_{\cL(\cA)} \|z\|_{\cL(\cA)}^*\\
&\leq \|z\|_{\cL(\cA_\tau)}^*+ \frac{16\rho\tau}{\pi(1-\rho)} \|z\|_{\cL(\cA)}^*\,.
\end{align*}
Rearranging both sides of this inequality gives
\[
	\|z\|_{\cL(\cA)}^* \leq C_\tau^{-1} \|z\|_{\cL(\cA_\tau)}^*
\]
with $C_\tau =  1-\tfrac{16\rho\tau}{\pi(1-\rho)}$, completing the proof.
\end{proof}

\begin{lemma} For any $a,b \in \D_\rho$,
\begin{equation}\label{eq:hankel-op-bound}
	\|\Gamma_{\varphi_a}-\Gamma_{\varphi_b}\|_1 \leq \frac{2\rho}{1-\rho} |a-b|\,.
\end{equation}
\end{lemma}
\begin{proof}
	The Hankel operator for $\varphi_a(z)$ is given by the semi-infinite, rank one matrix
	\[
		(1-|a|^2)\left[\begin{array}{ccccc}
			1 & a & a^2 & a^3 & \cdots\\
			a & a^2 & a^3 & a^4 & \cdots\\
			a^2 & a^3 & a^4 & a^5 & \cdots\\
			\vdots & \ddots
		\end{array}\right] = (1-|a|^2) \left[\begin{array}{c} 1 \\ a\\ a^2\\a^3 \\ \vdots \end{array}\right]
		\left[\begin{array}{c} 1 \\ a\\ a^2\\a^3 \\ \vdots \end{array}\right]^T\,.
	\]
Let $\zeta_a = \sqrt{1-|a|^2} \left[\begin{array}{cccccc} 1 & a & a^2 & a^3 & \cdots \end{array}\right]^T$.  Note that $\zeta_a \in \ell_2$ with norm equal to $1$.  Also note that we have
\begin{equation}\label{eq:zeta-dot}
	\langle \zeta_a, \zeta_b \rangle =  \frac{\sqrt{1-|a|^2}\sqrt{1-|b|^2}}{1-\bar{a}b}\,.
\end{equation}
Then we have
\begin{subequations}
\begin{align}
\nonumber	\|\Gamma_{\varphi_a}-\Gamma_{\varphi_b}\|_1 &= \| \zeta_a \zeta_a^T - \zeta_b \zeta_b^T\|_1\\
	& =  \| \zeta_a (\zeta_a-\zeta_b)^T +  (\zeta_a-\zeta_b) \zeta_b^T\|_1\\
	\label{eq:tri-ineq} &\leq \| \zeta_a (\zeta_a-\zeta_b)^T \|_1 + \| (\zeta_a-\zeta_b) \zeta_b^T\|_1\\
	\label{eq:nuc-2-l2} &= 2 \| \zeta_a-\zeta_b\|_{\ell_2}\\
\label{eq:l2-dist-formula}	&=2 \sqrt{2}\sqrt{1 - \mathfrak{Re} \frac{\sqrt{1-|a|^2}\sqrt{1-|b|^2}}{1-\bar{a}b}}\\
	\nonumber&\leq \frac{2 \rho}{1-\rho} |a-b|
\end{align}
\end{subequations}
Here,~\eq{tri-ineq}  is the triangle inequality.~\eq{nuc-2-l2} follows because the nuclear norm of a rank one operator is equal to the product of the $\ell_2$ norm of the factors.~\eq{l2-dist-formula} follows from~\eq{zeta-dot}.  The final inequality follows from analyzing the taylor series of the preceding expression.
\end{proof}

% section dast_for_system_identification (end)

% \begin{subappendices}
% 
% \section{Proof of Caratheodory's Theorem} % (fold)
% \label{pf:caratheodory}
% 
% 
% 
% % section caratheodory (end)	
% \end{subappendices}