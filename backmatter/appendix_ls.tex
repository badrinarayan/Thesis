\chapter{Dual Polynomial Properties}
\label{apx:collection}

In this section, we collect useful results about the dual polynomial for line
spectral signals from previous work, which we use in our analysis in
Chapter~\ref{chap:linespect}. We also derive some corollaries for some of these
theorems which we use. In addition to Theorem~\ref{dual-stab}, we recall another
result in~\cite{cg_noisy} where the authors show the existence of a
trigonometric polynomial $Q_1$ that is linear in each $N_j$ which is also an
essential ingredient in our proof.

\begin{theorem}[Lemma 2.7 in~\cite{cg_noisy}]
\label{dual-lin}
For any $f_1, \ldots, f_k$ satisfying \eqref{min-sep} and any sign vector $v \in
\C^k$ with $|v_j|=1$, there exists a polynomial $Q_1 = \left<q_1, a(f)\right>$
for some $q_1 \in \C^n$ with the following properties:
\begin{enumerate}
\item For every $f \in N_j,$ there exists a numerical constant $C_a^1$ such that
\begin{equation}
\label{ca1}
|Q_1(f) - v_j(f-f_j)| \leq \frac{n}{2} C_a^1 (f-f_j)^2
\end{equation}
\item For $f \in F$, there exists a numerical constant $C_b^1$ such that
\begin{equation}
\label{cb1}
|Q_1(f)| \leq \frac{C_b^1}{n}.
\end{equation}
\end{enumerate}
\end{theorem}

We will also need the following straightforward consequence of the constructions
of the polynomials in Theorem \ref{dual-stab}, Theorem \ref{dual-lin}, and
Section \ref{sec:support}.
\begin{lemma}
\label{l1}
There exists a numerical constant $C$ such that the constructed $Q(f)$ in
Theorem \ref{dual-stab}, $Q_1(f)$ in Theorem \ref{dual-lin}, and $Q_j^\star(f)$
in Section \ref{sec:support} satisfy respectively
\begin{align}
\|Q(f)\|_1 &:= \int_0^1{| Q(f)| df} \leq \frac{C k}{n}\label{QL1}\\
\| Q_1(f)\|_1 &\leq \frac{C k}{n^2}\label{Q1L1}\\
\|Q_j^\star\|_1 & \leq \frac{Ck}{n}\label{QjL1}.
\end{align}
\end{lemma}
\begin{proof}
We will give a detailed proof of \eqref{QL1}, and list the necessary modifications for proving \eqref{Q1L1} and \eqref{QjL1}. The dual polynomial $Q(f)$ constructed in \cite{cg_exact12} is of the
form
\begin{eqnarray}
  Q \left( f \right) & = & \sum_{f_j \in T} \alpha_j K \left( f - f_j \right)
  + \sum_{f_j \in T} \beta_j K' \left( f - f_j \right) \label{formofQ}
\end{eqnarray}
where $K \left( f \right)$ is the squared Fej\'er kernel (recall that $m = (n-1)/2$)
\begin{eqnarray*}
  K \left( f \right) & = & \left( \frac{\sin \left( \left( \frac{m}{2} + 1
  \right) \pi f \right)}{\left( \frac{m}{2} + 1 \right) \sin \left( \pi f
  \right)} \right)^4
\end{eqnarray*}
 and for $n \geq 257$, the coefficients
$\alpha \in \C^k$ and $\beta \in \C^k$ satisfy \cite[Lemma 2.2]{cg_exact12}
\begin{eqnarray*}
  \left\| \alpha \right\|_{\infty} & \leq & C_\alpha\\
  \left\| \beta \right\|_{\infty} & \leq & \frac{C_\beta}{n} 
\end{eqnarray*}
for some numerical constants $C_\alpha$ and $C_\beta$. 
Using \eqref{formofQ} and triangle inequality, we bound $\|Q(f)\|_1$ as follows: 
\begin{eqnarray}
  \|Q(f)\|_1 &=& \int_0^1 \left| Q \left( f \right) \right| d f\nonumber\\
  &\leq & k \left\| \alpha \right\|_\infty \int_0^1 \left| K \left( f\right) \right| d f + k \left\| \beta \right\|_\infty \int_0^1 \left| K' \left( f\right) \right| d f\label{Qbdgeneral1}\\
  & \leq & C_\alpha k \int_0^1 \left| K \left(f\right) \right| d f + \frac{C_\beta}{n} k \int_0^1 \left|K'(f)\right|d f \label{Q1bd},
 \end{eqnarray}
 
To continue, note that $\int_0^1 | K ( f) | d  f = \int_0^1 | G ( f) |^2 d
 f =: \|G(f)\|_2^2$ where $G ( f)$ is the Fej\'er kernel, since $K(f)$ is the squared
Fej\'er kernel. We can write
\begin{eqnarray}
  G ( f)  =  \left( \frac{\sin \left( \pi \left( \frac{m}{2} + 1 \right) f
  \right)}{\left( \frac{m}{2} + 1 \right) \sin ( \pi f)} \right)^2
   =  \sum_{l = - m / 2}^{m / 2} g_l e^{- i 2 \pi f l}\label{expressionG}
\end{eqnarray}
where $g_l = \left( \frac{m}{2} + 1 - | l | \right) / \left( \frac{m}{2} + 1
\right)^2$. Now, by using Parseval's identity, we obtain
\begin{eqnarray}
 \int_0^1 |K(f)| df & = & \int_0^1 | G ( f) |^2 d f
  =  \sum_{l = - m / 2}^{m / 2} | g_l |^2\nonumber\\
  & = & \frac{1}{\left( \frac{m}{2} + 1 \right)^4} \left( \left( \frac{m}{2}
  + 1 \right)^2 + 2 \sum_{l = 1}^{m / 2} \left( \frac{m}{2} + 1 - l \right)^2
  \right)\nonumber\\
  & = & \frac{1}{\left( \frac{m}{2} + 1 \right)^4} \left( \left( \frac{m}{2}
  + 1 \right)^2 + 2 \sum_{l = 1}^{m / 2} l^2 \right)\nonumber\\
  & \leq & \frac{C}{n}\label{bdKf}
\end{eqnarray}
for some numerical constant $C$ when $n = 2 m + 1 \geq 10$.

Now let us turn our attention to $\int_0^1 | K' (
f) | d  f$. Since $K ( f) = G ( f)^2$, we have
\begin{eqnarray}
  \int_0^1 | K' ( f) | d  f  =  2\int_0^1 | G ( f) G' ( f) | d
   f
   \leq  2\| G ( f) \|_2 \| G' ( f) \|_2\label{eq:holder-int}
\end{eqnarray}
We have already established that $\| G ( f) \|_2^2 \leq C / {n}$ and we
will now show that $\| G' ( f) \|_2^2 \leq C' {n}$. Differentiating the
expression for $G(f)$ in \eqref{expressionG}, we get
\begin{eqnarray*}
  G' ( f) & = & -2 \pi i \sum_{l = - m / 2}^{m / 2} l g_l e^{- i 2 \pi f l}
\end{eqnarray*}
Therefore, by applying Parseval's identity again, we get
\begin{eqnarray*}
  \| G' ( f) \|_2^2 
  & = & 4 \pi^2 \sum_{l = - m / 2}^{m / 2} l^2 | g_l |^2\\
  & \leq &  \pi^2 m^2 \sum_{l = - m / 2}^{m / 2} | g_l |^2\\
  & \leq & C' n
\end{eqnarray*}
Plugging back into \eqref{eq:holder-int} yields
\begin{align}
\int_0^1 |K'(f)| df \leq C \label{bdK1f}
\end{align}
for some constant $C$. Combining \eqref{bdK1f} and \eqref{bdKf} with \eqref{Q1bd} gives the desired result in \eqref{QL1}.



The dual polynomial $Q_1(f)$ is also of the form \eqref{formofQ} with coefficient vectors $\alpha_1$ and $\beta_1$, which satisfy \cite[Proof of Lemma 2.7]{cg_noisy}
\begin{align*}
\|\alpha_1\|_\infty \leq \frac{C_{\alpha_1}}{n},\\
\|\beta_1\|_\infty \leq \frac{C_{\beta_1}}{n^2}.
\end{align*}
Combining the above two bounds with \eqref{Qbdgeneral1}, \eqref{bdK1f} and \eqref{bdKf} gives the desired result in \eqref{Q1L1}.

The last polynomial $Q_j^\star$ also has the form \eqref{formofQ} with coefficient vectors $\alpha^\star$ and $\beta^\star$. According to \cite[Proof of Lemma 2.2]{granda2}, these coefficients satisfy
\begin{align*}
\|\alpha^\star\|_\infty \leq {C_{\alpha_\star}},\\
\|\beta_\star\|_\infty \leq \frac{C_{\beta_\star}}{n},
\end{align*}
which yields \eqref{QjL1} following the same argument leading to \eqref{QL1}. 

\end{proof}

Using Lemma \ref{l1}, we can derive the estimates we need in the following lemma.
\begin{lemma}
\label{l4}
Let $\nu = \hat{\mu} - \mu$ be the difference measure. Then, there exists numerical constant $C>0$ such that
\begin{align}
\label{qv}\left| \int_0^1 Q(f) \nu(df) \right| &\leq \frac{C k \tau}{n}\\
\label{q1v}\left| \int_0^1 Q_1(f) \nu(df) \right| &\leq \frac{C k \tau}{n^2}\\
\label{qjv} \left| \int_0^1 Q_j^\star(f) \nu(df) \right| & \leq \frac{Ck\tau}{n}.
\end{align}
\end{lemma}
\begin{proof}
Let $Q_0 = \langle q_0, a(f) \rangle $ be a general trigonometric polynomial associated with $q_0 \in \C^n$. Then,
\begin{align*}
\left|\int_0^1 Q_0(f) \nu(df) \right| 
& = \left|\int_0^1 \langle q_0 , a(f) \rangle  \nu(df) \right|\\
& = \left|\langle q_0,  \int_0^1  a(f)  \nu(df) \rangle\right|\\
& = \left|\langle q_0, e \rangle\right|\\
& = \left|\langle Q_0(f), E(f) \rangle\right|\\
& \leq \vnorm{Q_0(f)}_1 \vnorm{E(f)}_\infty\,.
\end{align*}
Here we use Parseval's identity in the second to last step and H\"{o}lder's inequality in the last inequality. Then, the result follows by using Lemma~\ref{l1} and \eqref{errbd}.
\end{proof}
% 
% We also need the following consequence of the optimality condition of AST
% from~\cite[Lemma 2]{btr12}:
% \begin{prop}\label{pro:optimality}
% \begin{align}
% \tau \vnorm{\hat{x}}_\A \leq \tau \vnorm{x^\star}_\A + \langle w, \hat{x} - x^\star \rangle
% \end{align}
% \end{prop}
