\chapter{Conclusion and Future Work} %(fold)
\label{chap:conclusion}

The convex optimization perspective on problems in signal processing and systems
holds a lot of promise. Our work forms the rudiments of continuous sparse
recovery which underlies a number of estimation problems in parametric signal
processing, systems theory and machine learning. In fact, using the atomic norm
framework introduced by~\cite{crpw}, we were able to revisit classical problems
and empirically and theoretically demonstrate performance favorable to and
comparable to the state-of-art.

The primary contribution of the thesis is a novel way to analyze regularization
with convex penalties for the case of continuous dictionaries that occur
naturally in several problems. In fact, we show that the recovery performance is
more tied to the target we wish to estimate and not the arbitrary coherence of
the atomic set. This turns the attention away from global properties of
dictionaries which answer the question ``Which dictionaries enable sparse
recovery using relaxations?'' to properties of signals. We employ a local
coherence condition characterized by minimum separation between frequencies for
line spectral signals which answers the question ``Which target signals are
recoverable using convex relaxations?''.

However, we have only scratched the surface of what is possible by continuing
this line of inquiry. In~\cite{cs_otg}, the authors extended our work to show
that using the atomic norm framework, we could achieve continuous compressed
sensing using Fourier measurements. Our theoretical analysis of the optimal
performance for continuous sparse recovery is heavily tied to Fourier
measurements and properties of trigonometric polynomials derived in
~\cite{CandesGranda,cg_noisy}. Future work should explore abstract conditions
when we can guarantee similarly optimal rates.

Algorithmically, the success of the method depends on whether we can efficiently
compute atomic norms. We showed that we can sometimes get exact semidefinite
characterizations and provided scalable algorithms. We also show that
discretization approximates the solution while it does not yield as sparse a
solution as the exact approach. An interesting line of work is showing how to
bootstrap an approximate solution from a fast discretization algorithm and
incrementally increase the accuracy of the solution. 

Analysis of the convergence of greedy algorithms provides yet another fruitful
direction for scaling the performance of atomic norm regularization. While there
is already some interesting work on employing greedy methods for atomic
norms~\cite{tewari2011greedy,raogreedy}, the proof of convergence depends upon
global restricted smoothness property. It would be interesting to pursue
improvements to this assumption, using arguments similar to our thesis.

% \section{Publications}
% 
% This dissertation is a result of a collaboration with Prof. Benjamin Recht, and
% postdoctoral students, Dr. Gongguo Tang and Dr. Parikshit Shah at the Wisconsin
% Institute of Discovery. Below, we provide a list of publications that composes
% much of this work.
% 




% chapter conclusion (end)