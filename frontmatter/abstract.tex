Many high dimensional phenomena observed in applications are simple and can be
approximated by a small combination of a potentially infinite number of building
blocks or atoms. It is possible to estimate such simple objects robustly from a
limited number of noisy measurements. Atomic norm regularization proposed in
this thesis is a convex penalty that can be used for deriving efficient
estimators of such high dimensional structures in a large number of cases.

This thesis provides a general approach to regularization using an atomic norm
penalty which unifies previous literature on high dimensional statistics. We
will revisit two fundamental problems in signal processing and systems theory --
line spectral estimation and system identification, which are classically
treated as nonlinear parameter estimation problems. We will see that a convex
approach proposed in this thesis can provide a principled way of tackling these
problems and provide optimal theoretical guarantees in the presence of noise. In
contrast, parametric approaches often need to estimate the number of atoms or
the model order and need heuristics to robustify nonlinear estimation.

The approach in this thesis can be thought of as a generalization of the Lasso
estimator for handling continuous infinite dimensional sparse recovery problems.
For the problem of line spectral estimation, I will provide efficient algorithms
based on an exact semidefinite characterization of the proposed estimator and
also show that discretization provides a scalable alternative to approximate the
solution for a number of problems.

