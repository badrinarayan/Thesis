\chapter{Introduction}

\begin{itemize}
  \item Structured High Dimensional Data are ubiquitous. The concept of simple models unifies them.
  \item Model Selection is NP-HARD. Convex Relaxations help.
  \item Atomic Norm is the natural convex relaxation.
  \item Interested in exact recovery and stable recovery.
\end{itemize}

We live in a world of data abundance. Due to advancements in data collection,
and massive storage capabilities, we now have a dizzying amount of high
dimensional data to analyze. The task of the data scientist is to infer a
simple model to describe the data. Fortunately, many naturally occurring
phenomena often have a simple structure. This allows us to efficiently
represent them using a few atoms or features, even if they are observed in a
high ambient dimension. The challenge is to exploit the simplicity of these
objects and recover them robustly from a few linear measurements. This can be a
formidable task even for simple instances of the problem.
%This can be thought of as a model selection or a feature selection problem.
% Problems involving estimation of high dimensional parameters under corrupted and limited linear measurements are ubiquitous.


For instance, imagine a data series that can be described by a linear model, in
terms of a large number of potential predictor variables. A simple description
of the data series expresses it in terms of a sparse combination of predictor
variables. Our task of choosing the ``right'' model for the data can be cast as
a combinatorial optimization problem with the objective of finding the smallest
subset of predictor variables that describe the data well. There does not
appear to be an efficient algorithm for this as there are an exponential number
of subsets to choose from. In fact, this selection problem is provably
\textsc{NP-Hard}\footnote{Informally, this means that it is at least as hard as
a large number of well known problems in complexity theory, widely believed to
have no algorithm that can solve them in a time proportional to any polynomial
function of the number of observations.}~\cite{Natarajan95}. This does not
however preclude the possibility of designing efficient algorithms that work on
\emph{most} instances. In fact, there is now a large body of literature on the
theoretical understanding of the surprising success of convex relaxation
methods in efficiently solving the sparse recovery problem most of the time.

The idea of treating simple objects as a sparse combination of atoms from a
possibly infinite dictionary unifies several related problems in sparse
recovery and approximation. Simple objects may be expressed as a sparse linear
combination of a few basic atoms drawn from a possibly infinite dictionary of
atoms. For instance, a sparse vector in high dimension is a combination of a
few canonical unit vectors. A low rank matrix is a combination of a few rank-1
matrices. A signal with a finite discrete spectrum is a combination of a few
frequencies.  In \cite{crpw}, the authors propose using ``atomic norm'' as a
convex heuristic to recover simple models from linear measurements. The atomic
norm may be thought of as a convex proxy to the combinatorial objective
function that arises naturally in sparse recovery problems.


Contributions and Organization of the Thesis.