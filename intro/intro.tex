%!TEX root = ../dissertation.tex
\chapter{Introduction}

\todo{TODO: Reorganize like this. First state our contribution. \emph{then}
discuss previous work.}
We live in a world of data abundance. Due to advancements in data collection,
and massive storage capabilities, we now have a dizzying amount of high
dimensional data to analyze. The task of the data scientist is to infer a simple
model to describe the data. Fortunately, many naturally occurring phenomena
often have a simple structure. This allows us to efficiently represent them
using a few atoms or features, even if they are observed in a high ambient
dimension. The challenge is to exploit the simplicity of these objects and
recover them robustly from limited linear measurements. This can be a formidable
task even for simple instances of the problem.

For example, imagine a data series composed of $n$ measurements that can be
described by a linear model, in terms of a large number $p \gg n$ of potential
predictor variables or features. Finding the linear model from these limited
measurements is a hopelessly underdetermined problem. However, when the linear
model is \emph{simple} and it is known a priori that only a small subset of the
features are actually active, the problem of determining the linear model is no
longer ill posed. A simple description of the data series expresses the data
series in terms of a few active features and corresponding feature weights. Our
task of choosing the ``right'' model for the data can be cast as a combinatorial
\emph{feature selection} problem with the objective of finding the smallest
subset of features that describe the data well. This is called the ``small n,
large p'' problem in statistics and such datasets are ubiquitous - For example,
biologists want to characterize samples in terms of a small subset of a large
number of genes from limited measurements. Machine learning practioners want to
infer the active features from a large feature space with a relatively small
number of measurements.

A naïve algorithm would explore all the possible subsets of features till we
find a solution. There does not appear to be an efficient algorithm for this as
there are an exponential number of subsets to choose from. In fact, this
selection problem is provably \textsc{NP-Hard}\footnote{Informally, this means
that it is at least as hard as a large number of well known problems in
complexity theory, widely believed to have no algorithm that can solve them in a
time proportional to any polynomial function of the number of
observations.}~\cite{Natarajan95}. This does not however preclude the
possibility of designing efficient algorithms that work on \emph{most}
instances. In fact, there is now a large body of literature on the theoretical
understanding of the surprising success of convex relaxation methods in
efficiently solving the sparse recovery problem most of the time.

The idea of using the $\ell_1$ norm of the feature weights (instead of the hard
combinatorial problem of minimizing sparsity of feature weights) as a convex
heuristic for model or feature selection, and the robustness of $\ell^1$
penalties to outliers and noise was known to several early practitioners in
Geophysics and
seismology~\cite{claerbout:robust,taylor:deconvolution,levy:spike-train,santosa}.
This was introduced in statistics as a sparsity inducing regularization
in~\cite{tibshirani96} and in Signal Processing as a means of exact and robust
decomposition by~\cite{chen98} and subsequently studied by Donoho and his
coworkers~\cite{donoho:huo,spark}. Since the seminal publication of
~\cite{CRT06,meinshausen:variable-selection}, which rigorously established the exact recovery and model selection properties the study of convex penalties for
feature selection has been a subject of intense research. This theory was
extended to other high dimensional structures including group sparse
vectors\cite{group:lasso}, low rank matrices~\cite{recht07} and cut
matrices~\cite{cut:goemans}.

%This can be thought of as a model selection or a feature selection problem.
% Problems involving estimation of high dimensional parameters under corrupted and limited linear measurements are ubiquitous.

The unifying theme in different structures in high dimension is some notion of
simplicity like sparsity for vectors or rank for matrices. Our work builds on a
recent publication~\cite{crpw} which discusses the notion of simple objects as a
sparse combination of atoms from a possibly infinite dictionary. This unifies
several related problems in sparse recovery and approximation. Simple objects
may be expressed as a sparse linear combination of a few basic atoms drawn from
a possibly infinite dictionary of atoms. For instance, a sparse vector in high
dimension is a combination of a few canonical unit vectors. A low rank matrix is
a combination of a few rank-1 matrices. A signal with a finite discrete spectrum
is a combination of a few frequencies. The authors propose using ``atomic norm''
as a convex heuristic to recover simple models from linear measurements. The
atomic norm may be thought of as a convex proxy to the combinatorial objective
function that arises naturally in sparse recovery problems.

Furthermore, the atomic norm framework allows us to naturally extend convex
relaxations to work on infinite dimensional objects which cannot be handled
satisfactorily using standard finite dimensional Lasso. In this thesis, I
discuss my joint work with Gongguo Tang, Parikshit Shah and Prof. Ben Recht on
denoising using atomic norms and the application of these ideas to revisit two
fundamental signal processing problems -- line spectral estimation and system
identification. We will also discuss efficient algorithms for these problems.

Line Spectral Estimation involves estimating frequencies and amplitudes from a
limited number of noisy measurements. This is an extremely well studied problem
in signal processing and there are a number of classical algorithms dating back
to Prony's technique. Surprisingly, using convex methods we shall discuss in
this thesis, we can outperform these classical techniques and also theoretically
characterize the performance of our algorithms.

The goal of system identification is to infer a low order system from a limited set of measurements. Using the convex methods discussed in the thesis, we can develop a simple algorithm for system identification. We see that it compares favorably in terms of prediction errors to the popular subspace ID method.
 
%  \\ \todo{Summarize
% Coherence, RIP. Emphasize Local Recovery and Off grid. Emphasize flexibility.
% Say problems like System ID and Line Spectral Estimation are not even
% expressible in this framework. Rephrase everything in terms of feature weights
% and use the term signal if possible.}

\section{Contributions and Organization} % (fold)
\label{sec:contributions}

In Chapter~\ref{chap:ast}, we will look at the atomic norm framework in depth
and see how it unifies many linear inverse problems in high dimensional
statistics. We extend this framework and introduce a general regularized estimator using the atomic norm penalty which we call Atomic Norm Soft Thresholding (AST). This may be thought of as an infinite dimensional version of Lasso. We establish universal properties of the estimator and indicate when accelerated convergence rates are possible. The choice of the regularization parameter for AST depends upon extremal properties of the dual atomic norm of noise, and we examine general techniques for estimating the regularization parameter.

In Chapter~\ref{chap:linespect}, we apply our atomic norm soft thresholding
estimator to denoise line spectral signals. We show consistency of estimation
for all signals using our universal slow rate. By exploiting properties of a
dual certificate constructed by Candes and Granda, we show that we can achieve
accelerated convergence rates when the frequencies in the line spectral signal
are well separated. As is the case for coherent designs using Lasso, we show
that this is nearly minimax optimal. This result may be thought of as a local
version of coherence. Although our dictionary is highly coherent, as long as the
signal we wish to recover is composed of relatively incoherent frequencies, it
is possible to recover it robustly. 

We also show that the frequencies localized by AST tend to be near the true
frequencies. We also show with extensive experiments that our proposal
outperforms classical line spectral estimation algorithms.

We turn our attention to the System Identification problem in
Chapter~\ref{chap:sysid} and show our technique can be adapted for this problem.
We describe how to setup an estimator for several kinds of linear measurements.
For the special case of frequency samples, we are able to derive finite sample
guarantees on the $\mathcal{H}^2$ prediction error of the transfer function
using AST.

Finally, in Chapter~\ref{chap:algos}, we look at algorithms for AST. It turns
out that we can efficiently solve AST as long as we have a reasonably efficient
algorithm to test membership in atomic norm ball. For the case of fourier
measurements, we show that atomic norm balls can be characterized by a
semidefinite program. While the positive case is classical, the general case is
a non-trivial extension. We describe how we can develop a fast parallelizable
algorithm for the SDP using Alternating Directions Method of Multipliers (ADMM).
We also show an alternative efficient discretized version of AST which can be
used in the absence of semidefinite characterizations and in general for large
problem sizes. This boils down to solving a Lasso problem on a grid. We study
the convergence of the Lasso solution which provides justification for discretization and Lasso on a grid as a general strategy.

% section contributions (end)